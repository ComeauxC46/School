%% LyX 2.1.4 created this file.  For more info, see http://www.lyx.org/.
%% Do not edit unless you really know what you are doing.
\documentclass[english]{article}
\usepackage{mathptmx}
\usepackage{helvet}
\usepackage{courier}
\usepackage[T1]{fontenc}
\usepackage[latin9]{inputenc}
\usepackage{geometry}
\geometry{verbose,tmargin=1in,bmargin=1in,lmargin=1in,rmargin=1in,headheight=0in,headsep=0in}
\pagestyle{empty}
\usepackage{babel}
\usepackage[unicode=true]
 {hyperref}

\makeatletter

%%%%%%%%%%%%%%%%%%%%%%%%%%%%%% LyX specific LaTeX commands.
%% Because html converters don't know tabularnewline
\providecommand{\tabularnewline}{\\}

%%%%%%%%%%%%%%%%%%%%%%%%%%%%%% User specified LaTeX commands.
\date{}

\makeatother

\begin{document}
\begin{center}
\textbf{\large{}CSCE 221 Assignment 5 Cover Page}\\
\bigskip{}

\par\end{center}

First Name~~~~~~~~~~~~Chris~~~~~~~~~~Last Name
~~~~~~~Comeaux~~~~~~~~~UIN~~~~622006681~~~~~~~~~~\bigskip{}


User Name ~~~~~~~~cmc236~~~~~~~~~~~~~~~~~~~~~E-mail
address~~~~~~~cmc236@tamu.edu~~~~~~~~~~~~~~~~~~~~~~~\medskip{}


Please list all sources in the table below including web pages which
you used to solve or implement the current homework. If you fail to
cite sources you can get a lower number of points or even zero, read
more on Aggie Honor System Office website: \texttt{\href{http://aggiehonor.tamu.edu/}{http://aggiehonor.tamu.edu/}}\medskip{}
\medskip{}


\noindent \begin{flushleft}
\begin{tabular}{|c|c|c|}
\hline 
Type of sources  & ~~~~~~~~~~~~~~~~~~~~~~~ & ~~~~~~~~~~~~~~~~~~~~~~~~\tabularnewline
 &  & \tabularnewline
\hline 
People & Peer Teacher & Dr. Teresa Leyk\tabularnewline
 &  & \tabularnewline
\hline 
Web pages (provide URL)  & http://www.cplusplus.com/ & http://stackoverflow.com/\tabularnewline
 &  & \tabularnewline
\hline 
Printed  & Programming Principles and Practices using C++ & \tabularnewline
 & by Bjarne Stroustrup & \tabularnewline
\hline 
Other Sources  & http://regexr.com/ & \tabularnewline
 &  & \tabularnewline
\hline 
\end{tabular}
\par\end{flushleft}

\medskip{}
\medskip{}


\noindent I certify that I have listed all the sources that I used
to develop the solutions/codes to the submitted work.

\noindent \emph{On my honor as an Aggie, I have neither given nor
received any unauthorized help on this academic work}.

\bigskip{}
\bigskip{}


\begin{tabular}{cccccc}
Your Name  & Chris &  & Comeaux & Date  & 4/4/2016\tabularnewline
\end{tabular}

\pagebreak{}


\section*{Assignment 5 Description}

In programming assignment 5 the students had to create a program that
could read in a csv file, parse the data using regex, hash the data
into a hash table, and then finally write the updated data to another
csv file. In this case, the input file was a file containing the name,
email, UIN, and a quiz grade of a student. The first part of the program
extracts the UIN and grade from input.csv and hashes the data by using
the UIN as a hash value and the grade as a value. Next, the program
reads in another csv file (roster.csv) and extracts the name, email,
and UIN from the file. It then searches the hash table using the UIN
to see if a grade exists for that student. If a grade exists, the
program will write the name, email, UIN, and grade to an output.csv.
If the grade does not exist, the program will only output the name,
email, and UIN. The output file should look just like roster.csv only
with the grades added to the students who took the quiz.


\section*{Data Structures and Algorithms}

\textbf{}%
\begin{minipage}[t]{1\columnwidth}%
\textbf{HashTable: }This is a user defined structure that was used
to describe a hash table; it has one member function and one helper
function. HashTable is simply just a STL vector of singly linked list
and uses the chaining method to deal with collisions. It also contains
a reside funtion that calls vector reside. This member function was
not necessary, however it cleans up the implementation in the main
file. It also has a helper function that is just the hash function
defined as UIN\%size.$\newline$

\textbf{SinglyLinkedList: }SinglyLinkedList is a very simple class
that describes a singly linked list only for the use for the chaining
method. It is made up of a sequence of SListNodes. It is made up of
2 SListNode pointers, head and tail, that points to the first and
last node respectfully. It has few member functions to insert a new
node at the end of the sequence, to remove all the nodes(used in destructor),
to search for a key return the key's value, and a function to check
if the list is empty. It is a friend class of SListNode so it is also
able to access its private members. $\newline$

\textbf{SListNode: }SListNode is a class that describes a node in
a SinglyLinkedList. It has a key, a value, and a pointer to the next
node. It has one member function that returns the next node. %
\end{minipage}


\section*{Description of Input and Output Data}

The input are 2 csv files that have 3-4 fields in each which are name,
email, UIN, and grade (for the input.csv only). The output is also
a csv file with the same 4 fields, however the grade field in not
filled in for every line. A few assumptions were make about the input
data. For example, an assumption was made that the names did not include
any middle names and were of the format (Name Name). Another one is
that is no type of error in the input files (i.e. wrong UIN or grade
format ). The program does not receive any input from the user so
no assumptions were imposed on the program, however, the program does
assume that 2 input files named, input.csv and roster.csv, are in
the directory with the program.


\section*{Testing}

To test for correctness, I first started by checking all of my regular
expressions. I would search each individually and output the matches
to the console to make sure they were matching the correct fields.
Once this was done, I completed writing the rest of the program. Once
I got program to terminate and produce the desired output, I started
to study the hash table to make sure it was working correctly. First,
I checked to see if the vector was re-sized to the correct size by
outputting the size of variable at ever step in the loop that calculated
the input file size. Next, I made sure that the hash function was
distributing the data as evenly as possible. To do this I inserted
a size accumulator into the SinglyLinkedList class and every time
a UIN was inserted I updated the size. I then looped through the whole
vector and output each list size to the console (2 was the largest
size). Finally, I compared both input files to the output file to
make sure no data was changed in any way.


\section*{C++ Features and Standard Library Classes}

I did not use any generic programming features in my program, however
I did use some object oriented features. For example, I created an
object to represent a hash table, a linked list, and linked list nodes.
Also, I used abstraction to hide private members and hide the implementation
of linked list behind the hash table class. I used 2 standard library
classes. I used the STL vector to create the hash table. Also, I use
the STL Regex class to create my regular expressions and extract the
fields from the csv files.


\section*{Running Time for Hashing Algorithms}

\begin{minipage}[t]{1\columnwidth}%
\textbf{Hash Function: }The running time function for the hash function
would be f(n) = 1 which is O(1).

\textbf{Search: }The running time function to search would be f(n)
= n+1 which is O(n). However, since the linked lists are so small
(max size = 2) they actually run in constant time so O(1).

\textbf{Insert: }The running time function to insert data into the
has table would be f(n) = 2 which is O(1).

\textbf{Destructor: }The running time for the desctructor (remove\_all
in a linked list) would be f(n) = 2n+2 which is O(n).

\textbf{Resize:} The running time for the reside funtion would be
f(n) = 1 which is O(1).%
\end{minipage}


\section*{Conclusion}

The purpose of this assignment was to teach students about hash tables
and the chaining method for dealing with collisions. Also, it thought
students about csv files and how to use regex to parse csv's. This
assignment additionally built on students knowledge of linked lists
and expanded their understanding of the implementations of linked
lists. This assignment will help students with future projects using
hash tables. 
\end{document}
